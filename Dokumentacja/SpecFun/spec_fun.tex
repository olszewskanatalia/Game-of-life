\documentclass[a4paper]{article}
\usepackage[utf8x]{inputenc}
\usepackage[T1]{fontenc}
\usepackage[MeX]{polski}
\usepackage{amssymb}
\usepackage{graphicx}
\usepackage{fancyhdr}
\usepackage{lastpage}
\usepackage{xcolor}

\pagestyle{fancy}
\fancyhf{}
\cfoot{ \thepage \hspace{1pt} / \pageref{LastPage}}
\lhead{Specyfikacja funkcjonalna}
\rhead{\texttt{GameOfLife}}


\begin{document}

\begin{titlepage}
	\begin{center}
		\vspace*{5cm}

	        \Huge
        	\textbf{Specyfikacja funkcjonalna automatu kom\'orkowego}

        	\vspace{1cm}
	        \Huge
        	\texttt{GameOfLife}: Gra w \.zycie Johna Conwaya

    		\vspace{1.5cm}

	        \large
		Aleksandra Michalska, Natalia Olszweska

        	\vfill

	        \vspace{3cm}

		\large 09.03.2021
	\end{center}
\end{titlepage}

\tableofcontents
\newpage


\section{Opis og\'olny}
\subsection{Nazwa programu}
\quad Nazwa programu to \texttt{\textit{"GameOfLife"}}.
\subsection{Wst\k{e}p teoretyczny}
\quad Gra w \.zycie Johna Conwaya jest automatem kom\'orkowym, czyli systemem sk\l{}adaj\k{a}cym si\k{e} z pojedynczych kom\'orek. Ka\.zda taka kom\'orka znajduje si\k{e} w jednym ze sko\'nczonej liczby stan\'ow (mo\.ze by\'c martwa lub \.zywa). 
\subsection{Cel projektu}
\quad Program ma na celu wy\'swietlanie kolejnych generacji gry w \.zycie przy u\.zyciu konsoli systemowej. Program mo\.ze dzia\l{}a\'c zar\'owno w trybie interaktywnym jak i wsadowym. Wybrane obrazy generowane przez program zapisywane mog\k{a} by\'c do pliku o rozszerzeniu graficznym.
\subsection{Cel dokumentu}
\quad Dokument ma na celu przybli\.zenie ko\.zystania z programu jego u\.zytkownikowi docelowemu.
\subsubsection{U\.zytkownik docelowy}
\quad Program jest powszechnie dost\k{e}pny oraz dedykowany jest dla ka\.zdego u\.zytkownika.



\section{Opis funkcjonalno\'sci}
\subsection{Mo\.zliwo\'sci programu}
\quad Program mo\.ze dzia\l{}a\'c w dw\'och trybach: \texttt{\textbf{step-by-step}} oraz \texttt{\textbf{fast}}. U\.zytkownik mo\.ze wybra\'c ile generacji obrazu chce zobaczy\'c oraz kt\'ore z nich nale\.zy zapisa\'c do pliku graficznego. Dost\k{e}pna jest opcja ustawienia trybu planszy 

\subsection{Argumenty wywo\l{}ania programu}
\quad Do poprawnego dzia\l{}ania programu potrzebne jest podanie na wej\'sciu wszystkich parametr\'ow podanych poni\.zej (istnieje mo\.liwo\'s\'c wyboru alternatywnych opcji). Program \texttt{\textit{GameOfLife}} akceptuje nast\k{e}puj\k{a}ce argumanty wywo\l{}ania:
\begin{itemize}
	\item \textbf{\texttt{ -in filein.txt}} nazwa pliku z danymi wej\'sciowymi
	\item \textbf{\texttt{ -out fileout.txt}} nazwa pliku do kt\'orego zapisywana b\k{e}dzie ko\'ncowa generacja programu
	\item \textbf{\texttt{ -n 7 }} ilo\'s\'c generacji do wy\'swietlenia
	\item \textbf{\texttt{ -s(o5 || f5) }}:
		\begin{itemize}
			\item \textbf{\texttt{ -s o5}} "save one" - zapisuje \texttt{5}- t\k{a} generacj\k{e} obrazu do pilku graficznego o rozszerzeniu .bmp (program sam przypisuje nazw\k{e} obrazowi)
			\item \textbf{\texttt{ -s f5}} "save first" - zapisuje pierwsze \texttt{5} obraz\'ow do plik\'ow graficznych o rozszerzeniach .bmp(program sam przypisuje nazwy obrazom)

		\end{itemize}
	\item \textbf{\texttt{ -m(sbs || fast)}}:
		\begin{itemize}
			\item \textbf{\texttt{ -m sbs}} "step-by-step mode" - tryb krok po kroku; u\.zytkownik naciskaj\k{a}c dowolny klawisz poza klawiszem \texttt{\textbf{f}} przechodzi do kolejnej generacji. 
				Istnieje mo\.zliwo\'s\'c przej\'scia z trybu \texttt{\textbf{sbs}} do trybu \texttt{\textbf{fast}} naciskaj\k{a}c klawisz \texttt{\textbf{f}}.
			\item \textbf{\texttt{ -m fast}} "fast mode" - tryb szybki; kolejne generacje wy\'swietlaj\k{a} si\k{e} automatycznie.
		\end{itemize}
	\item \textbf{\texttt{ -how(Ms || Mf || Ns || Nf)}}:
		\begin{itemize}
			\item \textbf{\texttt{ M }} liczba s\k{a}siad\'ow okre\'slana za pomoc\k{a} s\k{a}siedztwa Moore'a
			\item \textbf{\texttt{ N }} liczba s\k{a}siad\'ow okre\'slana za pomoc\k{a} s\k{a}siedztwa von Neumanna
			\item \textbf{\texttt{ s }} "sphere world" \'swiat dzia\l{}aj\k{a}cy jak sfera tzn. kom\'orki nie mog\k{a} spada\'s\'c z brzeg\'ow
			\item \textbf{\texttt{ f }} "flat world" \'swiat dzia\l{}ajacy jak p\l{}aska plansza tzn. kom\'orki mog\k{a} spada\'s\'c z brzeg\'ow\\

				

				Przyk\l{}adowo: \texttt{\textbf{-how Ms}} oznacza sferyczny \'swiat i s\k{a}siedztwo Moore'a.
		\end{itemize}

\end{itemize}

\subsection{Jak korzysta\'c z programu?}
\quad Aby uruchomi\'c program u\.zytkownik powinien utworzy\'c plik wej\'sciowy z danymi (jak opisano w sekcji "Dane wej\'sciowe") i nast\k{e}pnie, w terminalu, wybra\'c parametry uruchomienia (jak opisano w sekcji "Argumenty wywo\l{}ania programu").

Przyk\l{}adowe uruchomienie programu:
\begin{center}
	\texttt{\textbf{./game -in input.txt -out output.txt -n 20 -s o4 -m fast -how Nf}}
\end{center}
oznacza, \.ze:
\begin{itemize}
	\item dane czytane b\k{e}d\k{a} z pliku o nazwie input.txt
	\item wynikowe dane zostan\k{a} zapisane do pliku output.txt
	\item wy\'swietlone zostanie 20 generacji
	\item obraz o numerze 4 zostanie zapisany w postaci pliku graficznego o rozszerzeniu .bmp
	\item wybrany zosta\l{} tryb \texttt{\textbf{fast}}
	\item wybrane zosta\l{}o s\k{a}siedztwo von Neumanna i \'swiat w postaci plaskiej planszy
\end{itemize}


\section{Format danych i struktura plik\'ow}
\subsection{Dane wej\'sciowe}
\quad Dane wej\'sciowe s\k{a} przekazywane do programu w pliku tekstowym o rozszerzeniu .txt . W pilku powinny znajdowa\'c si\k{e} nast\k{e}puj\k{a}ce dane: 
\begin{itemize}
	\item \texttt{w} ilo\'s\'c wierszy $W = \{ w \in \mathbb{Z} : 3 \leq w \leq 30 \} $
	\item \texttt{k} ilo\'s\'c kolumn $K = \{ k \in \mathbb{Z} : 3 \leq k \leq 30 \} $
	\item wype\l{}nienie ka\.zdej kom\'orki: 0 (kom\'orka martwa) lub 1 (kom\'orka \.zywa)
		
		Przykladowe dane z pliku wejsciowego, wype\l{}niaj\k{a}ce tabel\k{e} o 4 wierszach i 4 kolumnach:
		\begin{table}[h!]
		\begin{center}
			\begin{tabular}{c c c c}
				4 & 4         \\
				0 & 0 & 1 & 0 \\
				0 & 1 & 0 & 1 \\
				1 & 0 & 0 & 0 \\
				0 & 1 & 1 & 0 \\
			\end{tabular}
			{\color{gray}\caption{Przyk\l{}adowe dane wej\'sciowe}}
			
			\label{table:tab1}
		\end{center}
		\end{table}

		Generuj\k{a} poni\.zszy obraz pocz\k{a}tkowy:

		\begin{figure}[h]
			\centering
			\includegraphics{obraz}
		\end{figure}
\end{itemize}

\subsection{Dane wyj\'sciowe}
\quad Dane wyj\'sciowe z programu zapisywane s\k{a} w postaci pliku tekstowego o rozszerzeniu .txt w takiej samej postaci jak dane w pliku wej\'sciowym (patrz: Tabela 1).

W zale\.zno\'sci od u\.zytych przy wywolaniu programu parametr\'ow, zapisywane mog\k{a} by\'c tak\.ze wybrane generacje programu do pliku graficznego o rozszerzeniu .bmp.



\section{Scenariusz dzia\l{}ania programu}
\subsection{Scenariusz og\'olny}
\quad Scenariusz og\'olny zak\l{}ada brak ingerencji u\.zytkownika w trakcie pracy programu po jego uruchomieniu.
\begin{enumerate}
	\item Tworzenie pliku wej\'sciowego
	
		U\.zytkownik tworzy poprawny plik wje\'sciowy z danymi (patrz: sekcja 3.1).
	\item Uruchomienie
	
		U\.zytkownik uruchamia program przy pomocy konsoli systemowej podaj\k{a}c parametry:
	
	\begin{center}
		\texttt{\textbf{./game -in input.txt -out output.txt -n 10 -s o2 -m fast -how Ms}}
	\end{center}
	\item Koniec pracy programu
	
		Po zako\'nczeniu pracy programu utworzony zostaje plik \texttt{output.txt} z danymi ostatniej generacji oraz obraz drugiej generacji.
\end{enumerate}
\subsection{Scenariusz szczeg\'o\l{}owy}
\quad Scenariusz szczeg\'o\l{}owy zak\l{}ada uruchomienie programu w trybie \texttt{\textbf{step-by-step}}.
\begin{enumerate}
	\item Tworzenie pliku wej\'sciowego	

		U\.zytkownik tworzy poprawny plik wje\'sciowy z danymi (patrz: sekcja 3.1).
	\item Uruchomienie
	
		U\.zytkownik uruchamia program przy pomocy konsoli systemowej podaj\k{a}c parametry:

        \begin{center}
                \texttt{\textbf{./game -in input.txt -out output.txt -n 10 -s f2 -m sbs -how Ms}}
        \end{center}
	\item Praca programu w trybie \texttt{\textbf{step-by-step}}

		U\.zytkownik wy\'swietla kolejne generacje programu wybieraj\k{a}c dowolny klawisz poza klawiszem \texttt{\textbf{f}}.
	\item Zmiana trybu pracy programu
	
		Znajduj\k{a}c si\k{e} w trybie \texttt{\textbf{step-by-step}}, po wy\'swietleniu kilku generacji, u\.zytkownik przechodzi do trybu \texttt{\textbf{fast}} wybieraj\k{a}c klawisz \texttt{\textbf{f}}. U\.zytkownik nie mo\.ze zmieni\'c ponownie trybu pracy programu. 
	\item Koniec pracy programu

        	Po zako\'nczeniu pracy programu utworzony zostaje plik \texttt{output.txt} z danymi ostatniej generacji oraz obrazy pierwszych dw\'och generacji.
\end{enumerate}

\subsection{Scenariusz w przypadku b\l{}\k{e}dnego uruchomienia}
\subsubsection{Nieprawid\l{}owy plik z danymi wej\'sciowymi}

\begin{enumerate}
	\item Tworzenie pliku wej\'sciowego
		U\.zytkownik tworzy plik z niepoprawnymi danymi np. brak podanej ilo\'sci kolumn lub wierszy, nieprawid\l{}owe dane tabeli etc. 
	\item Uruchomienie
		U\.zytkownik pr\'obuje uruchomi\'c program przy u\.zyciu niepoprawnie stworzonego pliku.
	\item Koniec pracy programu
		Program nie daje si\k{e} uruchomi\'c, wypisuje odpowiedni komunikat b\l{}\k{e}du i ko\'nczy prac\k{e}.
\end{enumerate}
\subsubsection{Nieprawid\l{}owe argumenty wywo\l{}ania}
\begin{enumerate}
	\item Tworzenie pliku wej\'sciowego
                U\.zytkownik tworzy poprawny plik wje\'sciowy z danymi (patrz: sekcja 3.1).
	\item Uruchomienie
		U\.zytkownik pr\'obuje uruchomi\'c program przy pomocy konsoli systemowej podaj\k{a}c przyk\l{}adowe nieprawid\l{}owe parametry:
	\begin{center}
                \texttt{\textbf{./game -in input.txt -out output.txt -n 10 -s f2 -m sbs -incorrect parameter}}
        \end{center}
	\item Koniec pracy programu
                Program nie daje si\k{e} uruchomi\'c, wypisuje odpowiedni komunikat b\l{}\k{e}du i ko\'nczy prac\k{e}.
\end{enumerate}

\subsection{Komunikaty b\l{}\k{e}d\'ow}
\quad W zale\.zno\'sci od rodzaju b\l{}\k{e}du wy\'swietlany b\k{e}dzie odpowiedni komunikat:
\begin{enumerate}
	\item W przypadku nieprawid\l{}owych danych w pliku wej\'sciowym wy\'swietlany b\k{e}dzie komunikat: \texttt{FileError} i parca programu zostanie przerwana.
	\item W przypadku podania nieprawid\l{}owych parametr\'ow wywo\l{}ania programu (nieistniej\k{a}cy parametr, zbyt ma\l{}o paarametr\'ow, zbyt du\.zo parametr\'ow), wy\'swietlany b\k{e}dzie komunikat: \texttt{ParametersError} i praca programu zostanie przerwana.
\end{enumerate}

\end{document}
